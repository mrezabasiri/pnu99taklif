\documentclass[14pt]{beamer}
\usetheme{Warsaw}
\usepackage[utf8]{inputenc}
\author{presentation with Beamer
page 45 and 46 book Introduction to Automata Theory,
Formal Languages and
Computation
from the student mohamadreza basiri payamnoor tehran
shomal}
\title{in the name of god}
%\setbeamercovered{transparent} 
%\setbeamertemplate{navigation symbols}{} 
%\logo{} 
%\institute{} 
%\date{} 
%\subject{} 
\begin{document}
Fill in the Blanks

 1. Grammar consists of four touplesSet of non-terminals......set of production rule, and
.......
 
 
 2. According to the Chomsky hierarchy, there are .......types of grammars.
 
 
3. Type 1 grammar is called........
 
 
 4. Type 2 grammar is called......
 
 
 5. According to the Chomsky hierarchy, regular grammar is type.......grammar.
 
 
 6. All languages are accepted by.......
 
 
 7. The machine format of context-free language is.......
 
 
 8. Linear bounded automata is the machine format of ........
 
 
 9. The machine format of type 3 language is.......
 
 
 10. Grammar where production rules are in the format  is....... grammar
 
 
 11. In a context-free grammar at the left hand side, there is........non-terminal.
 
 
 12. Type 3 language is called........
 
 
 13. an bn cn is an example of.......language in particular.
 
 
 14. The grammar S → aSb/A, A →Ac/c is an example of .......grammar in particular.
 
 
 15. The grammar S → Abc/ABSc, BA → AB, Bb → bb, A → a is an example of grammar........in particular.
 
 
 16. The grammar A → aA/bB/a/b, B → bB/b is an example of........grammar in particular.


 17. The language a*(a + b)b* is an example of .......language in particular.
 
 
 Answer the question above sentence
  
  
  1. set of terminals, start symbol 
  
  
  2. Four 
  
  
  3. Context-sensitive grammar
  
  
 4. Context-free grammar 
 
 
 5. Three 6. Turing machine
 
 
 7. Push down automata 
 
 
 8. Context-sensitive grammar
 
  
 9. Finite automata
 
 
10. Context-sensitive 


11. Single 



12. Regular expression


13. Context-sensitive 


14. Context-free 


15. Context-sensitive


16. Regular grammar 


17. Type 4 


Find the languages generated by the following grammars


a) S → aSb/A, A →Ac/c
 
 
 b) S → aSb/aAb, A → Ac/e


 c) S → aSb/aAb, A → bA/b
 
 
 d) S → S1/S2, S1 → 0S11/0A, A → 0A/, S2 → 0S21/B1, B → B1/e
 
 
 e) S → AB/CD, A → aA/a, B → bB/bC, C→ cD/d, D → aD/AD
 
 
 Justify your answer for this.
 
 
 f) S → AA, A → BS, A → b, B → SA, B→ a
 
 
 g) E → E + E| E menhha E| E * E| E/E|id
 
 Construct a grammar for the following languages.
a) L = {tohi}
 
 
 b) L = (a, b)*, where all ‘a’ appears before ‘b’
 
 
 c) L = (a, b)*, where all ‘b’ appears before ‘a’
 
 
 d) L = (a, b)*, where there are equal number of ‘a’ and ‘b’
 
 
 e) L = (a, b)*, where ab and ba appear in an alternating sequence.
 
 
 f) L = (a, b)*, where the number of ‘b’ is one more than the number of ‘a’


 g) L = (a, b)*aa(a, b)*


Construct a grammar for the following languages.


a) L = ambn
, where m not equal to n.


 b) L = axbycz, where y = x + z
 
 
 c) L = axbycz, where z = x + y
 
 
 d) L = axbycz, where x = y + z
 
 
 e) L = Set of all string over a, b containing aa or bb as substring
 
 
 f) L = Set of all string over a, b containing at least two ‘a’
 
 
 g) L = Set of all string over 0, 1 containing 011 as substring
 
 
 Construct a grammar for the following languages.
 
 
 a) { anbn| n 0} u {cmdm | m0 }


b) { anbn| n 0} u {ambm | m0 }
 
 c) { axbycz, where x = y + z } u { L = axbycz, where z = x + y}
 
 
 Construct a grammar for the following languages and find the type of the grammar in particular
 
 
 a) L = (0 + 1)* 11 (11)*
 
 
 b) L = (Set of all string of ‘a’, ‘b’ beginning and ending with ‘a’ )
 
 
 c) L = a2n + 1, where n > 0

\begin{frame}
\titlepage
\end{frame}

%\begin{frame}
%\tableofcontents
%\end{frame}

\begin{frame}{•}
 
\end{frame}

\end{document}